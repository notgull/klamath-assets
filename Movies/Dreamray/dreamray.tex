\documentclass[11pt,a4paper,oneside]{memoir}  % http://www.ctan.org/tex-archive/macros/latex/contrib/memoir
\usepackage[english]{babel}
\usepackage[utf8]{inputenc}
\usepackage{enumitem}  % http://www.ctan.org/tex-archive/macros/latex/contrib/enumitem

\newlength{\drop}  % Without this, the title page will not compile correctly.
% To avoid using drop, see: http://wiki.lyx.org/LyX/UsingMemoirInLyX

\chapterstyle{demo2}  % See p92 of the Memoir manual.

\pagestyle{myheadings}

\setlength{\parindent}{0pt}

\renewcommand{\printtoctitle}[1]{\centering\Large\bfseries Acts}  % Set the title of the contents page.
% \renewcommand{\printtoctitle}[1]  % Remove the title from the contents page entirely.

\pagenumbering{gobble} % Remove page numbers until told otherwise.

% Various title pages may be used with the memoir package.  The one below is from ``Some Examples of Title Pages'' (Peter Wilson) at http://www.ctan.org/tex-archive/info/latex-samples/TitlePages.

% Set up the title page.
\newcommand*{\titleGM}{\begingroup% Gentle Madness title page style
   \drop = 0.1\textheight
   \vspace*{\baselineskip}
   \vfill
   \hbox{%
       \hspace*{0.2\textwidth}%
       \rule{1pt}{\textheight}
       \hspace*{0.05\textwidth}%
       \parbox[b]{0.75\textwidth}{
           \vbox{%
               % Main title of the play
               \vspace{\drop}{\noindent\HUGE\bfseries Dreamray}\\
               %\vspace{\drop}{\noindent\HUGE\bfseries Title of the play \\[0.5\baselineskip] over two lines}\\
               % Subtitle of the play
               [2\baselineskip]{\huge\itshape A short spectacle}\\
               %[2\baselineskip]{\Large\itshape Subtitle of the play \\[0.5\baselineskip] over two lines}\\
               % Author of the play
               [4\baselineskip]{\Large John Notgull}\par\vspace{0.5\textheight}
               %[4\baselineskip]{\Large First Author \\[0.5\baselineskip] Second Author \\[0.5\baselineskip] Third Author \\}\par\vspace{0.5\textheight}
               % Publisher and year of publication
               {\noindent \textbf{notgull} \\[0.5\baselineskip] \textbf{2021}}\\
               [\baselineskip]
           }% end of vbox
       }% end of parbox
   }% end of hbox
   \vfill
   \null
\endgroup}

\begin{document}

% Print out the title page.
\titleGM

\pagenumbering{roman} % Start numbering pages with Roman numerals (for the front matter).

% Print out the contents page, listing the acts of the play.
% You will need to run pdflatex twice before the page numbers show up.
\tableofcontents*
\clearpage

% Print out the characters page, listing the dramatis personae
% The starred form of \chapter prevents a chapter number (eg ``One'', ``Two'') being printed before each chapter title (eg ``Characters'', ``Act 1'').
\chapter*{CHARACTERS}
\begin{center}  % Centre the list of characters.  Comment out this line and \end{}center if centring is not desired.
\textbf{Klamath}, an optimist, unfortunately \\
\textbf{Dr. Chester}, a scientist without ethics
\vskip 1cm

\textbf{Scene}: Klamath and Chester's House.
\textbf{Time of action}: A minute or so.
\end{center}

% Print out a page with any additional authorial comments, notes on staging, or whatever.
\chapter*{PREFACE, NOTES, WHATEVER}

% Set up a description list to hold the paragraphs.  Increase the space between the list items, and set the left margin to 0.20cm
\begin{description}[itemsep=1ex,leftmargin=0.20cm]

% Precede each paragraph with an empty \item[].
\item[] This is meant to be a short little joke, with the two primary objectives of sharpening my skills in not only animation but also voice acting, and to illustrate two simple characters playing out a scene.

\item[] Thanks to Nathan for proofreading.

\end{description}

\clearpage
\pagenumbering{arabic}  % Start numbering pages with Arabic numerals (for the text of the play).

% Generate a running header with the title of the play.
\markright{\textsc{Dreamray}}

%%%%%%%%%%%%%%%%
\chapter*{ACT 1}
%%%%%%%%%%%%%%%%
% The \chapter* will prevent the the chapters (Acts) being listed in the table of contents, so we need to add them manually.
\addcontentsline{toc}{chapter}{Act 1}

% The starred form of \section prevents a section number (eg ``1.1'', ``2.3'') being printed before each section title (eg ``Scene 1'', ``Scene 2'').
\section*{\textit{SCENE 1}}
%\section*{\hfill\textit{SCENE 1}}  % Use this line instead if you want the Scene 1 heading shifted to the right edge of the page.

% Set up a description list to hold the dialogue of the scene.  Increase the space between the list items, and set the left margin to 1cm.
\begin{description}[itemsep=1ex,leftmargin=1cm]

% Where the scene or act begins with stage directions:
\item[] \hfill \\
\textit{Establishing shot of a two-story house in a neighborhood. The house is relatively plain, with blue paint and white windowsills, combined with a brown tile roof. A car is parked in the driveway: a rusted-over station wagon that looks like it's barely holding together. However, this contrasts with the houses next to it. The house to the (viewer's) left is plain and white; however, the house to the right appears to be a toaster of some kind. The sky is blue and the sun is shining.}

\item[] \hfill \\
\textit{Suddenly, a small smart-car pulls swiftly into the driveway. The driver is a man wearing a Carlos Santana-esque mustache and a yellow tie. Skid marks in the driveway indicate this isn't the first time he's done this.}

\item[] \hfill \\
\textit{The scene cuts to the inside of the house. Klamath is sitting at the kitchen table, opening mail. Klamath is a blob-person who looks similar to Marvin the Martian but wearing the Neighborhood Watch man's getup. He opens a letter, takes out the contents and reads them, before throwing his hands into the air and cheering.}

\item[KLAMATH] \textit{(whoops)} Yeah!

\item[] \hfill \\
\textit{Cut to the foyer. The door is thrown open as Chester steps inside.}

\item[CHESTER] What is up, Klamath!

\item[] \hfill \\
\textit{He begins approaching Klamath as Klamath begins to speak.}

\item[KLAMATH] Great news, Chester! I got accepted to Harvard!

\item[CHESTER] I've got better news! I created a new invention that will make us millionaires!

\item[] \hfill \\
\textit{As he speaks, Chester draws a ray gun from his jacket pocket}

\item[KLAMATH] What's that? A new gun?

\item[CHESTER] It's much more than that. You see, when you shoot someone with a normal gun, you have to deal with the mess of the body and the family and whatnot. With the DreamRay, it simply ruins their dreams, rendering them a shallow husk of a man. It's like murder, but better.

\item[KLAMATH] Cool, how does it work?

\item[CHESTER] I don't know if it does yet. Let's find out!

\item[] \hfill \\
\textit{Chester fires the DreamRay at Klamath. After being briefly enveloped in a pink aura, Klamath checks the letter again.}

\item[KLAMATH] Aww, I got accepted to Penn. \textit{(with anger)} Hey!

\textit{Chester giggles while Klamath jumps from the chair, snatches the DreamRay from his hands, and fires it back at him. The DreamRay transforms into a regular gun.}

\item[CHESTER] Well that's disapointing, now it's just a normal gu-

\item[] \hfill \\
\textit{Gunshot, followed by sudden blackness before credits roll.}

% Where the scene or act begins with dialogue, and no stage directions:
%\item[] \hfill

% Close the description list at the end of the scene.
\end{description}
\vskip 1cm  % Put a bit of space between this and the next scene heading.

\vskip 1cm


\end{document}
